\chapter{Specifikacija rada}

Ovaj rad za cilj ima teoretski opisati i prakti\v{c}nim primjerom testirati dva sli\v{c}na be\v{z}i\v{c}ne protokola, NFC (Near Field Communication) i BLE (Bluetooth Low Energy). Motivi za odabir ovakve teme uklju\v{c}uju relativnu , veliko podru\v{c}je primjene i razli\v{c}ite mogu\'{c} nosti koje pru\v{z}aju oba protokola. Me\dj utim, glavni motiv je sveprisutnost navedenih protokola jer danas gotovo svaki novi pametni telefon ima ugra\dj en NFC i BLE modul. Ako se uzme u obzir da je kori\v{s}tenje pametnog telefona postala svakodnevica gotovo polovice ljudi na svijetu (po izvje\v{s}\'{c} u ``Ericsson Mobility Report'' tvrtke Ericsson \cite{mobilityReport} 2015. godine se u svijetu se koristilo 3,4 milijardi pametnih telefona, a predvi\dj eno je da \'{c} e se do 2021. ta brojka popeti do \v{c}ak 6,4 milijardi), dolazi se do zaklju\v{c}ka da mobilne aplikacije koje u svojim funkcionalnostima koriste NFC ili BLE protokol imaju potencijalno ogromno tr\v{z}i\v{s}te. Ipak, treba sa rezervom uzeti toliku brojku jer se oba protokola tek po\v{c}inju ugra\dj aviti u ve\'{c} inu novih pametnih telefona, dok su ih proteklih godina proizvo\dj a\v{c}i ugra\dj ivali samo u svoje najja\v{c}e i najskuplje modele.

Sli\v{c}nost protokola je u tome \v{s}to se oba koriste za be\v{z}i\v{c}nu komunikaciju kratkog dometa. Me\dj utim, tehnologija koja se koristi za implementaciju protokola je potpuno razli\v{c}ita. NFC za prijenos podataka koristi svojstva elektromagnetske indukcije, dok se kod BLE-a prijenos podataka ostvaruje preko radio valova. Samim time su svojstva protokola razli\v{c}ita (najbolji primjer je domet - NFC u praksi ima domet do 5 cm, a BLE do 10 metara) \v{s}to na kraju rezultira razli\v{c}itom primjenom u praksi. Upravo zato su protokoli komplementarni i zajedno se ugra\dj uju u pametne telefone jer zajedno mogu pru\v{z}iti rje\v{s}enje za gotovo sve potrebe u kratko dometnoj komunikaciji (razina prostorije). Naravno, razlog tome je i to \v{s}to su pametni telefon vrlo napredni ure\dj aji koji osim NFC i BLE modula imaju i GSM modul, modul za mobilni internet i WiFi modul, koji nisu uvijek optimalni za komunikaciju u kratkom dometu. Me\dj utim, kombinacija svih navedenih modula i mogu\'{c}nosti protokola koje implementiraju, \v{c}ini pametni telefon tako naprednim ure\dj ajem bez kojeg je \v{z}ivot modernog \v{c}ovjeka u 21. stolje\'{c}u nezamisliv.

Zbog svega navedenog, temeljna ideja ovog rada je implementirati oba protokola u sustav koji ima smisla i koji ima potencijala za\v{z}ivjeti na dana\v{s}njem tr\v{z}i\v{s}tu. Nastavak ovog poglavlja sadr\v{z}i opis sustava, aktivnosti koje su poduzete za ostvarivanje sustava te krajnji rezultat.

\section{Specifikacija sustava}

Glavna ideja sustava je kreirati mobilnu aplikaciju i administrativno su\v{c}elje koje bi trgova\v{c}ki lanci koristili za promociju proizvoda u svojim poslovnicama. Ideja je da trgova\v{c}ki lanaci preko internetskog su\v{c}elja kreiraju popuste za svoje proizvode u odabranim poslovinicama, a zatim kupci pomo\'{c}u mobilne aplikacije ostvaruju kreirane popuste. Korisni\v{c}ko iskustvo je zami\v{s}ljeno tako da korisnik prilikom ulaza u poslovnicu, pomo\'{c}u pametnog telefona sa instaliranom aplikacijom te NFC i BLE modulom, skenira NFC naljepnicu koja aplikaciji daje informaciju u koju je poslovnicu korisnik u\v{s}ao. Mobilna aplikacija zatim dohva\'{c}a konfiguraciju te poslovnice poslovnice sa poslu\v{z}itelja, te je ta akcija je prikazana na slici ~\ref{fig:skeniranjeNaljepnice}.

\begin{figure}[!htbp]
	\begin{center}
 \includegraphics[height=12cm,keepaspectratio=true]{nfc_sken}
 \caption{Prikaz procesa skeniranja NFC naljepnice (1), zahtjeva za konfiguracijom poslovnice (2) i dobivanje konfiguracje poslovnice (3).}
 \label{fig:skeniranjeNaljepnice}
	\end{center}
\end{figure}

Kada je aplikacija dobila konfiguraciju po\v{c}inje sa skeniranjem okoline, s ciljem nala\v{z}enja BLE ure\dj aja. Proizvodi na akciji imaju u svojoj neposrednoj blizini BLE ogla\v{s}iva? te korisniku koji prolazi pokraj police od proizvoda, ukoliko ima upaljenu aplikaciju, prona\dj eni popust postaje vidljiv u aplikaciji. Tada, ukoliko se odlu\v{c}i na iskori\v{s}tavanje popusta, kreira zahtjev za kodom popusta. Zahtjev je vezan za korisnikov ure\dj aj (zbog za\v{s}tite od zloupotrebe - svaki ure\dj aj mo\v{z}e jedan popust ostvariti maksimalno jedan put) te korisnik dobiva kod za popust kojeg je, s ciljem ostvarivanja popusta, du\v{z}an prikazati na blagajni. Opisani postupci su prikazani na slici  ~\ref{fig:otkrivanjeBLEa}.

\begin{figure}[!htbp]
	\begin{center}
 \includegraphics[height=12cm,keepaspectratio=true]{ble_sken}
 \caption{Prikaz procesa otkrivanja BLE ogla\v{s}iva\v{c}a (1), zahtjev za kodom skeniranog popusta (2), dobivanje koda za popust (3) i prikazivanje koda na blagajni za kona\v{c}no ostvarivanje popusta (4).}
 \label{fig:otkrivanjeBLEa}
	\end{center}
\end{figure}

Za implementaciju opisanog sustava potrebne su slijede\'{c}e aktivnosti:
\begin{enumerate}
	\item Kreiranje web aplikacije sa su\v{c}eljem za poslovne subjekte
	\item Kreiranje API su\v{c}elja za komunikaciju mobilne aplikacije i poslu\v{z}itelja
	\item Konfiguriranje NFC naljepnica i BLE ogla\v{s}iva\v{c}a
	\item Kreiranje mobilne aplikacije
	
\end{enumerate}

Resursi potrebni za ostvarivanje aktivnosti uklju\v{c}uju:
\begin{enumerate}
	\item NFC naljepnice
	\item BLE ogla\v{s}iva\v{c}i
	\item Pametni telefon s integriranim NFC i BLE modulom
	\item Poslu\v{z}itelj za pohranjivanje internetske aplikacije i baze podataka
\end{enumerate}


\section{Rezultati}

Rezultat ovog rada je teoretska obrada dva sli\v{c}na be\v{z}i?na protokola za prijenos podataka te sustav koji objedinjuje i NFC i BLE protokol te uz pomo\'{c}u njihovih specifi\v{c}nosti korisnicima pru\v{z}a novo i druga\v{c}ije iskustvo u obavljanju kupovine. Prakti\v{c}ni dio rada uklju\v{c}uje u potpunosti funkcionalnu internetsku i mobilnu aplikaciju. Internetska aplikacija se sastoji od dva dijela:

\begin{itemize}
	\item Su\v{c}elje za trgova\v{c}ke lance
	\begin{itemize}
		\item Implementirano dodavanje i ure\dj ivanje poslovnica
		\item Implementirano dodavanje popusta za odre\dj eni proizvod i povezivanje popusta sa odgovaraju\v{c}im ogla\v{s}iva\v{c}em
		\item Implementirano upravljanje popustima i pregledavanje iskori\v{s}tenih popusta
	\end{itemize}
	\item API su\v{c}elje
	
	\begin{itemize}
		\item Omogu\'{c}ava komunikaciju poslu\v{z}itelja i mobilne aplikacije
	\end{itemize}
\end{itemize}

Funkcionalnosti mobilne aplikacij uklju?uju:
\begin{enumerate}
	\item Skeniranje NFC naljepnica
	\item Tra\v{z}enje BLE ogla\v{s}iva\v{c}a u okolini
	\item Komunikacija sa poslu\v{z}iteljem
\end{enumerate}

U nastavku rada su opisane specifi\v{c}nosti NFC i BLE protokola, specifi\v{c}nosti tehnologija i alata pomo\'{c}u kojih je sustav kreiran, detaljan opis implementacije sustava te na poslijetku usporedba i evaluacija opisanih protokola.





