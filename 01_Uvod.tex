\chapter{Uvod}

Tehnologija je sve prisutna u dana\v{s}njem svijetu. Ne postoji grana ljudske djelatnosti koja u zadnjih 30 godina nije redefinirana dolaskom elektroni\v{c}kih ure\dj aja. Danas se u sve ugra\dj uju elektroni\v{c}ki sklopovi koji pobolj\v{s}avaju i pro\v{s}iruju funkcionalnosti ure\dj aja. Sve vi\v{s}e i vi\v{s}e ure\dj aja dobiva prefiks ``pametni'', \v{s}to ozna\v{c}ava da ure\dj aj sadr\v{z}i neku vrstu mikroprocesora koji u pozadini izvr\v{s}ava neku logiku i time unaprije\dj uje ure\dj aj. \v{S}to postoij vi\v{s}e takvih ure\dj aja, to je ve\'{c}a potreba za protokolima pomo\'{c}u kojih \'{c}e ure\dj aji komunicirati sa drugim ure\dj ajima. Umre\v{z}avanjem ure\dj aja se ponovno pro\v{s}iruju funkcionalnosti istih i dolazi se do novih mogu\'{c}nosti i podru\v{c}ja primjene.

Ovaj rad za cilj ima evaluirati i u prakti\v{c}nom primjeru implementirati dva sli\v{c}na be\v{z}i\v{c}na komunikacijska protokola, NFC (Near Field Communication) i BLE (Bluetooth Low Energy). Glavni motiv odabira ovih protokola je njihova sveprisutnost jer danas gotovo svaki novi pametni telefon ima ugra\dj en NFC i BLE modul. Ako se uzme u obzir da je kori\v{s}tenje pametnog telefona postala svakodnevica gotovo polovice \v{c}ovje\v{c}anstva (po izvje\v{s}\'{c}u ``Ericsson Mobility Report'' tvrtke Ericsson \cite{mobilityReport} 2015. godine se u svijetu koristilo 3,4 milijardi pametnih telefona, a predvi\dj eno je da \'{c}e se do 2021. ta brojka popeti do \v{c}ak 6,4 milijardi), dolazi se do zaklju\v{c}ka da mobilne aplikacije koje u svojim funkcionalnostima koriste NFC ili BLE protokol imaju potencijalno ogromno tr\v{z}i\v{s}te. Ipak, treba sa rezervom uzeti toliku brojku jer se oba protokola tek po\v{c}inju ugra\dj aviti u ve\'{c}inu novih pametnih telefona, dok su ih proteklih godina proizvo\dj a\v{c}i ugra\dj ivali samo u svoje najja\v{c}e i najskuplje modele.

Sli\v{c}nost protokola je u tome \v{s}to se oba koriste za be\v{z}i\v{c}nu komunikaciju kratkog dometa. Me\dj utim, tehnologija koja se koristi za implementaciju protokola je potpuno razli\v{c}ita. NFC za prijenos podataka koristi svojstva elektromagnetske indukcije, dok se kod BLE-a prijenos podataka ostvaruje preko radio valova. Samim time su svojstva protokola razli\v{c}ita (najbolji primjer je domet - NFC u praksi ima domet do 5 cm, a BLE do 10 metara) \v{s}to na kraju rezultira razli\v{c}itom primjenom u praksi. Upravo zato su protokoli komplementarni i zajedno se ugra\dj uju u pametne telefone jer zajedno mogu pru\v{z}iti rje\v{s}enje za gotovo sve potrebe u kratko dometnoj komunikaciji (razina prostorije). Naravno, razlog tome je i to \v{s}to su pametni telefon vrlo napredni ure\dj aji koji osim NFC i BLE modula imaju i GSM modul, modul za mobilni internet i WiFi modul, koji nisu uvijek optimalni za komunikaciju u kratkom dometu. Me\dj utim, kombinacija svih navedenih modula i mogu\'{c}nosti protokola koje implementiraju, \v{c}ini pametni telefon tako naprednim ure\dj ajem bez kojeg je \v{z}ivot modernog \v{c}ovjeka u 21. stolje\'{c}u nezamisliv.

Zbog svega navedenog, temeljna ideja ovog rada je implementirati oba protokola u sustav koji ima smisla i koji ima potencijala za\v{z}ivjeti na dana\v{s}njem tr\v{z}i\v{s}tu. Nastavak ovog poglavlja sadr\v{z}i opis sustava, aktivnosti koje su poduzete za ostvarivanje sustava te krajnji rezultat.
