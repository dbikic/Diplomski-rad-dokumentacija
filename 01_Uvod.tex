\chapter{Uvod}

Tehnologija je sveprisutna u dana�njem svijetu. Ne postoji grana ljudske djelatnosti koja u zadnjih 30 godina nije redefinirana dolaskom elektroni?kih ure?aja. Danas se u sve ugra?uju elektroni?ki sklopovi koji pobolj�avaju i pro�iruju funkcionalnosti ure?aja. Sve vi�e i vi�e ure?aja dobiva prefiks ``pametni'', �to ozna?ava da ure?aj sadr�i neku vrstu mikroprocesora koji u pozadini izvr�ava neku logiku i time unaprje?uje ure?aj. �to postoji vi�e takvih ure?aja, to je ve?a potreba za protokolima pomo?u kojih ?e ure?aji komunicirati sa drugim ure?ajima. Umre�avanjem ure?aja se ponovno pro�iruju njihove funkcionalnosti i dolazi do novih mogu?nosti i podru?ja primjene.

Ovaj rad za cilj ima evaluirati i u prakti?nom primjeru implementirati dva sli?na be�i?na komunikacijska protokola, NFC i BLE. Glavni motiv odabira ovih protokola je njihova sveprisutnost jer danas gotovo svaki novi pametni telefon ima ugra?en NFC i BLE modul. Ako se uzme u obzir da je kori�tenje pametnog telefona postala svakodnevica gotovo polovice ?ovje?anstva (po izvje�?u ``Ericsson Mobility Report'' tvrtke Ericsson \cite{mobilityReport} 2015. godine se u svijetu koristilo 3,4 milijardi pametnih telefona, a predvi?eno je da ?e se do 2021. ta brojka popeti do ?ak 6,4 milijardi), dolazi se do zaklju?ka da mobilne aplikacije koje u svojim funkcionalnostima koriste NFC ili BLE protokol imaju potencijalno ogromno tr�i�te. Ipak, treba sa rezervom uzeti toliku brojku jer se oba protokola tek po?inju ugra?ivati u ve?inu novih pametnih telefona, dok su ih proteklih godina proizvo?a?i ugra?ivali samo u svoje najja?e i najskuplje modele.

Sli?nost protokola je u tome �to se oba koriste za be�i?nu komunikaciju kratkog dometa. Me?utim, tehnologija koja se koristi za implementaciju protokola je potpuno razli?ita. NFC za prijenos podataka koristi svojstva elektromagnetske indukcije, dok se kod BLE-a prijenos podataka ostvaruje preko radio valova. Samim time su svojstva protokola razli?ita (najbolji primjer je domet - NFC u praksi ima domet do 5 cm, a BLE do 10 metara) �to na kraju rezultira razli?itom primjenom u praksi. Upravo zato su protokoli komplementarni i zajedno se ugra?uju u pametne telefone jer zajedno mogu pru�iti rje�enje za gotovo sve potrebe u kratko dometnoj komunikaciji (razina prostorije). Naravno, razlog tome je i to �to su pametni telefon vrlo napredni ure?aji koji osim NFC i BLE modula imaju i GSM modul, modul za mobilni internet i WiFi modul, koji nisu uvijek optimalni za komunikaciju u kratkom dometu. Me?utim, kombinacija svih navedenih modula i mogu?nosti protokola koje implementiraju, ?ini pametni telefon tako naprednim ure?ajem bez kojeg je �ivot modernog ?ovjeka u 21. stolje?u nezamisliv.

Zbog svega navedenog, temeljna ideja ovog rada je implementirati oba protokola u sustav koji ima smisla i koji ima potencijala za�ivjeti na dana�njem tr�i�tu. Nastavak ovog poglavlja sadr�i opis sustava, aktivnosti koje su poduzete za ostvarivanje sustava te krajnji rezultat.
