\chapter{Zaklju\v{c}ak}

Svrha ovog rada je usporediti NFC i BLE kao dva protokola be\v{z}i\v{c}ne komunikacije koji postaju sveprisutni u dana\v{s}njim pametnim telefonima. Iako su protokoli u su\v{s}tini razli\v{c}iti, njihovo podru\v{c}je primjene je isto - kratkodometna komunikacija. Kroz rad su opisani tehni\v{c}ki detalji i principi rada pojedinog protokola te je opisana njihova primjena u praksi.
Iako im primjena nije striktno vezana uz pametne telefone, ve\'{c}ina korisnika ovih protokola se susrela s njima kroz pametni telefon. 

Tr\v{z}i\v{s}te pametnih telefona i mobilnih aplikacija je puno potencijala jer danas gotovo svaki \v{c}ovjek posjeduje pametni telefon. Kori\v{s}tenje ovih protokola dodatno pro\v{s}iruje korisni\v{c}ko iskustvo mobilnih aplikacija i dodaje neke nove standarde u njihovom kori\v{s}tenju. Primjer toga je sustav napravljen kao prakti\v{c}ni dio ovog rada, koji je tako\dj er primjer inovacije u mobilnom ogla\v{s}avanju. Rezultat sustava su mobilna i internet aplikacija koje korisnicima pru\v{z}aju kompletno iskustvo upravljanja ogla\v{s}avanjem. No, to je samo jedan od primjera iskori\v{s}tavanja mogu\'{c}nosti pojedinog protokola, zajedno s mogu\'{c}nostima koje pru\v{z}aju pametni telefoni i internet.

Budu\'{c}nost industrije mobilnih aplikacija le\v{z}i upravo u ovakvim sustavima koji su sinteza razli\v{c}itih tehnologija te time korisnicima pru\v{z}aju novo i zanimljivo iskustvo u kori\v{s}tenju mobilnih aplikacija.
