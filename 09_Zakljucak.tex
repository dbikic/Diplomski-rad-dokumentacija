\chapter{Zaklju?ak}
Iako se radi o dva razli?ita protokola, ima ih smisla uspore?ivati jer im je danas naj?e�?a primjena vezana uz pametne telefone. Me?utim, najve?a razlika im je u dometu komunikacije (domet BLE-a je red veli?ine nekoliko metra a domet NFC-a red veli?ine nekoliko centimetra) pa im se shodno tome primjena razlikuje. Ovo poglavlje ?e 


 NFC is limited to a distance of approximately four centimeters while Bluetooth can reach over thirty feet. While it may seem that Bluetooth is superior in this regard, both Bluetooth and NFC technology have their advantages and disadvantages compared to one another and can work together to meet users? needs.
 
 
 android
web
sustav


ovaj sustav je primjenjiv u stvarnom svijetu,

mo�e se napraviti puno sustava koji pro�iruju korisni?ko iskustvo

budu?nost aplikacija ide u tom smjeru jer je korisniima dosadno koristiti sve isto

Of late, indoor location technology has grown over and beyond its outdoor counterpart. In fact, it currently plays a critical role in reinventing the mobile advertising, and app development industry. So much so that, it has now paved the way for a number of location-based , be it enhanced customer engagement, improved navigation or risk mitigation. Adding on to that, it is highly critical for businesses these days to have access to indoor location information that is accurate, and cost effective. So, which among the following widely used location technologies ? Wi-Fi, iBeacon, NFC, GPS, will your business benefit from?

This is one of the questions that keeps surfacing in most of our conversations with customers till date. Unfortunately, each of these technologies have their own limitations and businesses need to use the right combination of two or more based on their budget and what they are trying to achieve. We have already discussed about the basic differences between Wi-Fi and iBeacon (Bluetooth Low Energy) technology and how they work best when used together.
