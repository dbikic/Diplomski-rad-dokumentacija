\chapter*{Sa\v{z}etak}
\addcontentsline{toc}{chapter}{Sa\v{z}etak}  
Ovaj rad bavi se evaluacijom dva be\v{z}i\v{c}na kratkodometna komunikacijska protokola: Bluetooth Low Energy i Near Field Communication. Evaluacija pojedinog protokola sastoji se od opisivanja nastanka protokola, arhitekture i primjene u dana\v{s}njem svijetu. Kao prakti\v{c}ni dio rada je implementiran sustav koji koristi oba protokola za svoje funkcionalnosti. Sustav se sastoji od mobilne i internetske aplikacije i svrha mu je pru\v{z}iti platformu za upravljanje popusta trgova\v{c}kih lanaca. Korisni\v{c}ko iskustvo je zami\v{s}ljeno tako da kupac pametnim telefonom skenira NFC naljepnicu na ulazu poslovnice te tako inicira tra\v{z}enje popusta. Kada kupac do\dj e u neposrednu blizinu proizvoda na akciji mobilna aplikacija detektira signal od BLE ogla\v{s}iva\v{c}a (postavljen pokraj proizvoda). Tada se korisniku u aplikaciji prika\v{z}e popust kojeg mo\v{z}e aktivirati te tako dobiva kod kojeg prika\v{z}e na blagajni da popust i ostvari. Internetska aplikacija uklju\v{c}uje su\v{c}elje za upravljanje popustima i aplikacijsko programsko su\v{c}elje za komunikaciju mobilne aplikacije i baze podataka na serveru.


\vspace{1.5\baselineskip}

\noindent \textbf{Klju\v{c}ne rije\v{c}i:} Bluetooth Low Energy; Near Field Communication; internetska aplikacija; mobilna aplikacija; Android

\chapter*{Abstract}
\addcontentsline{toc}{chapter}{Abstract}  

This thesis evaluates two wireless short-range communication protocol's: Bluetooth Low Energy and Near Field Communication. Evaluation of each protocol includes descriptions of protocol's evolution, architecture, and usage in today's world. The practical part of the thesis includes the creation of a system that uses both protocols in its features. The system consists of mobile and web application and its purpose is to create a platform for managing retail chains product discounts. It does that by providing NFC tags at the store's entrance, which has to be scanned by an application installed on a customer's smartphone. The application connects to an application programming interface and gets the store's configuration and thus, the discount search begins. When a customer walks nearby a product with the discount, the application detects a signal that a BLE beacon emits (the beacon also has to be placed on the products shelf). The application presents the discount to the customer which he can activate to get a code. To claim the discount he shows the code to the cashier. The web application consists of a discount management system and an application programming interface which is used for communication of mobile application with a database on the web server.

\vspace{1.5\baselineskip}
\noindent \textbf{Keywords:} Bluetooth Low Energy; Near Field Communication; web application; mobile application; Android
